\section{Introduction}
\label{sec:intro}

Internet users must entrust private information to many different companies, but online security is often overlooked. Even large and technically advanced companies have lost sensitive information to data breaches. For example, Yahoo suffered two major breaches in the past few years. First, in 2013 attackers perpetrated the largest recorded data breach when they stole data from roughly one billion Yahoo accounts \cite{Thielman:2016}. Then in 2014, a separate attack compromised roughly 500 million Yahoo accounts \cite{Fiegerman:2016}. Neither of these breaches were discovered until 2016, meaning sensitive user information such as hashed passwords and security questions were stolen years before any user became aware.

These types of breaches are not unique to Yahoo \cite{Hunt:2016, Tabachnik:2016}. They exemplify a major issue with internet security: users must entrust their personal information to servers and companies that are not transparent and often fail to implement standard security methods. Most companies do not publish their full security practices, and even users who are responsible or knowledgeable about online security have no means to verify that their data is being handled correctly. Furthermore, the companies themselves are often unable to detect breaches promptly. As users register for an increasing number of services, the risks described above grow as well.

In spite of the difficulties and flaws of passwords, such as reused, weak, and difficult-to-remember passwords as well as constant breaches of improperly stored passwords, passwords are still unavoidable for users. Despite a push from both users and security experts, alternatives rarely catch on, especially in web authentication \cite{Herley:2011}. Given that passwords have not been eliminated, we focus on improving the security of user passwords rather than eliminating them entirely.
We propose Web Authentication without Transmitting or Storing User Passwords (WATSUP), a new application-layer protocol that provides users with an auditable, consistent, and easy way to log into web services without trusting any service to properly handle their login credentials. In this paper, we first discuss security risks, their solutions and drawbacks, and related work. Next, we propose the WATSUP protocol; then we discuss a proof-of-concept implementation; and finally, we discuss the advantages and disadvantages of WATSUP and propose future work.
